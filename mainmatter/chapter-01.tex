\chapter{Introduzione}

La Meccanica Applicata alle Macchine ha come obiettivo lo studio delle macchine o meglio dei sistemi meccanici.
Cerchiamo per prima cosa di chiarire il concetto di macchina:
\define{Macchina}{costruzione dell'uomo, il cui stato evolve nel tempo, atta al raggiungimento di un prefissato obiettivo.}

Dato che l'obiettivo principale dell'uomo è l'esecuzione di un determinato lavoro utile, la macchina può essere vista come: \emph{Sistema di organi disposti in modo tale da compiere lavoro di interesse industriale}.
L'esecuzione del lavoro comporta, di conseguenza, la presenza di forze applicate e trasmesse tra i vari organi della macchine e di spostamenti degli stessi.

Secondo un approccio classico le macchine si possono dividere in 2 classi principali (in base alla loro funzione):
    \begin{enumerate}
        \item \textbf{Macchine energetiche}:
            costituite allo scopo di ottenere determinate trasformazioni di energia.
            Esse si possono distinguere a loro volta tra:
            
            \begin{itemize}
                \item macchine \textbf{motrici}: trasformano l'energia di altra forma in energia meccanica
                \item macchine \textbf{generatrici}: trasformano l'energia meccanica in energia di altra forma
            \end{itemize}
        \item \textbf{Macchine operatrici}:
            costruite allo scopo di realizzare specifiche operazioni diverse dalla trasformazione di energia.
             È possibile formulare un'ulteriore distinzione delle diverse macchine operatrici per esempio in base
             al loro utilizzo (e.g. macchine utensile, macchine di sollevamento, macchine tessili, mecchine per lo scavo e sollevamento terra, robot, macchine da trasporto...)
    \end{enumerate}

A questa classificazione formulata secondo un \emph{approccio classico}, è preferito un secondo approccio, detto approccio metodologico.\\
La principale distinzione tra i due approcci risiede nella classificazione in base alle loro caratteristiche cinematiche e dinamiche (metodologico)
piuttosto che in base alla loro funzione (classico).

Secondo le premesse appena esposte è possibile, dunque, dare una nuova definizione di Macchina:

\define{Macchina}{La macchina è un  sistema meccanico costituito da più componenti o sottoinsiemi. Ciò che caratterizza il sistema è il modello matematico (i.e. insieme di equazioni che ne rappresentano il comportamento cinematico e dinamico in determinate condizioni).}
dove un \textbf{Sistema meccanico} consiste in un insieme di più componenti (solidi, deformabili, rigidi, fluidi, liquidi, aeriformi) collegati tra di loro da coppie cinematiche in modo da realizzare determinati moti relativi tra di loro e in modo da trasmettere determinate forze.
     
Il sistema in questione è caratterizzato dal fatto di essere separabile dall'ambiente mediante un confine di tipo fisico o concettuale attraverso il quale avvengono scambi di energia o di informazioni, e di essere costituito da più componenti interconnessi in modo da formare un'unica entità.
Note le proprietà dei sistemi meccanici, l'obiettivo fondamentale della meccanica applicata è, di conseguenza, la costruzione di un modello matematico (è possibile costruirne più di uno tra cui è da preferire il più semplice tra quelli che riescono a conservare le caratteristiche essenziali del sistema reale) che è possibile formulare tramite la cosiddetta operazione di sintesi.

L'operazione di sintesi consiste nell'utilizzo di una serie di procedure razionali per la scelta del tipo di sistema meccanico, del numero dei suoi membri e delle sue dimensioni.
La sintesi è suddivisa in tre fasi:
\begin{itemize}
    \item \textbf{sintesi di tipo}: scelta del tipo di sistema meccanico 
    \item \textbf{sintesi di numero}: definizione del sistema (e.g. numero di componenti e loro collegamento relativo)
    \item \textbf{sintesi dimensionale}: determinazione delle dimensioni degli organi del sistema
\end{itemize}


\begin{table}[!h]
\centering
\begin{NiceTabular}{R{0.33\columnwidth}Y{0.33\columnwidth}P{0.33\columnwidth}}
    \midrule
    {\scshape{\bfseries cinematica diretta}} & {\scshape{\bfseries problemi di analisi}} & {\scshape{\bfseries dinamica diretta}}\\
    Assegnato il modo del membro motore, studio del moto degli algtri membri del sistema (indipendentemente dalle forse che li generano). && Studio del moto dei membri generato dalle forze applicate o dalle condizioni iniziali diverse da quelle di equilibrio stabile\\
    \midrule
    {\scshape{\bfseries cinematica inversa}} & {\scshape{\bfseries problemi di analisi}} & {\scshape{\bfseries dinamica inversa}}\\
    Assegnato il moto di uno o di alcuni membri condotti del sistem determinare come deve muoversi il membro motore affinché il membro condotto sia quello desiderato && Assegnate le forze applicate al sistema e le leggi del moto del membro motore, determinare il valore della forza motrice da applicare affinché si ottenga il moto desiderato.\\
    \midrule
    {\scshape{\bfseries cinematica }}&{\scshape{\bfseries problemi di sintesi}}&{\scshape{\bfseries dinamica}}\\
    Determinare con procedure razionali le caratteristiche strutturali del sistema in grado di realizzare il moto desiderato && Determinare con procedure razionali le caratteristiche strutturali del sistema in grado di realizzare il moto desiderato
\end{NiceTabular}
\end{table}
